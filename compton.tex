\documentclass[landscape,a4]{myslides}
\usepackage{theorem}
\usepackage[dvips]{color}
\usepackage{calc}
\usepackage{ifthen}
\usepackage{epsfig}
%\usepackage{graphicx}
\usepackage{latexsym}
\usepackage{amssymb,amsmath}
\input{smacros}
\skiptrue
%----------------headers----------------------
\def\Lecturename{Logical Methods in Combinatorics, 236605-2009/10}  %xxx=ESSLLI'99
\def\Lecture{Lecture ?}      %xxx=Lecture

\begin{document}

\begin{slide}
\begin{center}
{%\large
\color{blue}
0-1 laws for slow growing classes
}
\\
\_\hrulefill
\end{center}

Our goal:
{\small
\begin{itemize}
\item
Define ``slow growing'' classes of structures (as opposed to ``fast growing'' which we shall not discuss)
\item
Find a probability measure on MSOL sentences that extends the usual measure (but is defined for any MSOL sentence)
\item
Study under which conditions a class has a 0-1 law with respect to this measure.
\end{itemize}
}
\end{slide}
 
\begin{slide}
\begin{center}
{%\large
% 17 lines
\color{blue}
Definitions
}
\\
\_\hrulefill

\end{center}
\small
\begin{itemize}
 \item Let $C$ be a class of labeled structures, and $C_n$ be the class of structures of size $n$ in $C$.
 \item Set $a_n = |C_n|$
 \item The \emph{exponential generating series} for $C$ is the formal power series $a(x)=\sum_{n=0}^\infty\frac{a_n}{n!}x^n$
 \item When considering $a(x)$ as an analytical function on $\mathbb{C}$ it makes sense to consider its radius of convergence $R$. If $R>0$ we say that $C$ is
a \emph{slow growing} class (since the faster the sequence $a_n$ grows, the smaller the radius of convergence).
\end{itemize}

\end{slide}

\begin{slide}
\begin{center}
{%\large
\color{blue}
The problem with the standard probability measure
}
\\
\_\hrulefill


\end{center}

\small
Recall that given a  sentence $\varphi$ we defined $\mu(\varphi)=\lim_{n\to\infty}\mu_n(\varphi)$ with $\mu_n(\varphi)=|\left\{\mathcal{M}\in C_n|\mathcal{M}\models\varphi \right\}| / |C_n|$.

However, the value of $\mu_n(\varphi)$ might constantly alternate values, causing $\mu(\varphi)$ to be undefined.

Example:

Let $\varphi$ be the following property of $G=(V,E)$: ``$|V|$ is even and $|E|=0$'' (expressable in SOL).

Obviously, $\mu_{2n}(\varphi) = 1$ and $\mu_{2n+1}(\varphi) = 0$
\end{slide}

\begin{slide}
\begin{center}
{%\large
\color{blue}
Extending the $\mu$ function
}
\\
\_\hrulefill
\end{center}
\small

We deal with a class of structures $C$ closed under disjoint unions and components, such that the egf $a(x)$ for $C$ has radius of convergence $R>0$ and $\lim_{x\to R}a(x)=\infty$.

Given $\varphi$, denote by $c(x)$ the egf corresponding to the class defined by $\varphi$.

For this case, define $\overline{\mu}(\varphi) = \lim_{x\to R} c(x)/a(x)$

Using standard calculus techniques it can be shown that if $\mu(\varphi)$ exists, then $\mu(\varphi)=\overline{\mu}(\varphi)$.

It remains to be shown that $\overline{\mu}(\varphi)$ is defined for \emph{every} MSO $\varphi$.
\end{slide}

\begin{slide}
\begin{center}
{%\large
\color{blue}
Two facts about constructions of structures
}
\\
\_\hrulefill
\end{center}
\small

By denoting $\mathcal{M}\approx_r\mathcal{N}$ we mean that the structues $\mathcal{M}, \mathcal{N}$ satisfy the same MSO sentences of quantifier rank at most $r$.

\emph{Fact 1}: If $\mathcal{M}_0\approx_r\mathcal{N}_0$ and $\mathcal{M}_1\approx_r\mathcal{N}_1$ then $\mathcal{M}_0\uplus\mathcal{M}_1\approx_r\mathcal{N}_0\uplus\mathcal{N}_1$

\emph{Fact 2}: For every $r$ and a similarity type there is an integer $s$ such that whenever $i,j\ge s$ and $\mathcal{M}$ is a structure of this similarity type, $i\cdot\mathcal{M}\approx_rj\cdot\mathcal{M}$

$i\cdot\mathcal{M}$ denotes the disjoint union of $i$ copies of $\mathcal{M}$.

The similarity type of $\mathcal{M}$ is the tuple of the arities of the relations over $\mathcal{M}$.

For example, the similarity type of a graph is $(2)$ since we have only one binary relation.

The similarity type of a colored graph is $(2,1)$ since we have the binary edge relation, and the unary color relation.

We prove both facts by showing that in an Ehrenfeuch-Fraisse game with $r$ rounds over the composite structures, the duplicator has a winning strategy.
\end{slide}

\begin{slide}
\begin{center}
{%\large
\color{blue}
Proof of fact 1
}
\\
\_\hrulefill
\end{center}
\small
\emph{Fact 1}: If $\mathcal{M}_0\approx_r\mathcal{N}_0$ and $\mathcal{M}_1\approx_r\mathcal{N}_1$ then $\mathcal{M}_0\uplus\mathcal{M}_1\approx_r\mathcal{N}_0\uplus\mathcal{N}_1$

The duplicator simply plays two games simultaneously. If the spoiler plays a move on $\mathcal{M}_0$, the duplicator replies with the move on $\mathcal{N}_0$  he would have played in the strategy proving $\mathcal{M}_0\approx_r\mathcal{N}_0$ .

Other moves of the spoiler are dealt with similarily.

Obvoiusly, after $r$ moves, the substructures on $\mathcal{M}_0$ and $\mathcal{N}_0$ are isomorphic, and so are the substructures on $\mathcal{M}_1$,$\mathcal{N}_1$.

The disjoint union of two isomorphic substructures gives an isomorphic composite stucture, and so the duplicator has won.
\end{slide}

\begin{slide}
\begin{center}
{%\large
\color{blue}
Proof of fact 2
}
\\
\_\hrulefill
\end{center}
\small
\emph{Fact 2}: For every $r$ and a similarity type there is an integer $s$ such that whenever $i,j\ge s$ and $\mathcal{M}$ is a structure of this similarity type, $i\cdot\mathcal{M}\approx_rj\cdot\mathcal{M}$

This is not a private case of fact 1, since we have a \emph{different number} of copies of $\mathcal{M}$ in the two structures.

Proceed with induction on $r$ (we prove the claim for \emph{all} similarity types ``at once'').

The spoiler takes the first move, by choosing a subset of (without loss of generality) $i\cdot \mathcal{M}$. We think of this move as adding a unary relation to the similary type of the structure (``coloring'' some vertices in a graph).

Problem: We can no longer think of the structure $i\cdot \mathcal{M}$ as a disjoint union of the same structure $\mathcal{M}$ since different copies might look different (has different ``color'' patterns).

Solution: Forget about $\mathcal{M}$, find another way to divide the structures into disjoint union of ``similar'' substructures.

\end{slide}

\begin{slide}
\begin{center}
{%\large
\color{blue}
Proof of fact 2 - cont.
}
\\
\_\hrulefill
\end{center}
\small
\emph{Fact 2}: For every $r$ and a similarity type there is an integer $s$ such that whenever $i,j\ge s$ and $\mathcal{M}$ is a structure of this similarity type, $i\cdot\mathcal{M}\approx_rj\cdot\mathcal{M}$

Given a similarity type and $r$, there is only a finite number $t$ of equivalence classes of $\approx_{r-1}$ for that similarity type.

We can think of the colored $i\cdot \mathcal{M}$ as a union of structures from all the equivalence classes.

Denote by $S_1^i,S_2^i,\dots,S_t^i$ the sets of components of $i\cdot \mathcal{M}$ belonging to each equivalence class, 
i.e. $i\cdot \mathcal{M} = \biguplus S_k^i$ and for all $k$, $S_k^i$ contains a number of equivalent substructures.

\end{slide}

\begin{slide}
\begin{center}
{%\large
\color{blue}
Proof of fact 2 - cont.
}
\\
\_\hrulefill
\end{center}
\small
\emph{Fact 2}: For every $r$ and a similarity type there is an integer $s$ such that whenever $i,j\ge s$ and $\mathcal{M}$ is a structure of this similarity type, $i\cdot\mathcal{M}\approx_rj\cdot\mathcal{M}$

Given a similarity type and $r$, there is only a finite number $t$ of equivalence classes of $\approx_{r-1}$ for that similarity type.

We can think of the colored $i\cdot \mathcal{M}$ as a union of structures from all the equivalence classes.

Denote by $S_1^i,S_2^i,\dots,S_t^i$ the sets of components of $i\cdot \mathcal{M}$ belonging to each equivalence class, 
i.e. $i\cdot \mathcal{M} = \biguplus S_k^i$ and for all $k$, $S_k^i$ contains a number of equivalent substructures.

\end{slide}


\begin{slide}
\begin{center}
{%\large
\color{blue}
Proof of fact 2 - cont.
}
\\
\_\hrulefill
\end{center}
\small

We can think of the spoiler's move as ``splitting`` each copy of $\mathcal{M}$ between some of the $S_k^i$'s.

The bottom line: For each $S_k^i$ the duplicator can create a similar $S_k^j$ such that one of two things hold:
\begin{enumerate}
 \item $S_k^i$ has the same number of substructures as $S_k^j$, so they're trivially isomorphic.
 \item $S_k^i$ and $S_k^j$ both contain \emph{a lot} of substructures, and we can use the induction hypothesis to show they're isomorphic.
\end{enumerate}

\end{slide}


\begin{slide}
\begin{center}
{%\large
\color{blue}
Proof of fact 2 - cont.
}
\\
\_\hrulefill
\end{center}
\small

\emph{Fact 2}: For every $r$ and a similarity type there is an integer $s$ such that whenever $i,j\ge s$ and $\mathcal{M}$ is a structure of this similarity type, $i\cdot\mathcal{M}\approx_rj\cdot\mathcal{M}$

To precisely define ''a lot``, let $s'$ be the constant guaranteed by the induction hypothesis for $r-1$ and the new similarity type.

Now define $s=s't$, and so $i,j\ge s't$. This means that even if coloring $\mathcal{M}$ only contributes to \emph{one} euivalence class $S_k^j$, the duplicator
can still fill every $S_k^j$ with at least $s'$ equivalent copies.

So, if after the spoiler's move, $S_k^i$ contains at most $s'$ substructures, the duplicator ensures the same number of subsutrcutres would be in $S_k^j$.

If there are more than $s'$ substructures in $S_k^i$, the duplicator ensures there are at least $s'$ substructures in $S_k^j$, and then uses the induction hypothesis on them.

In any case, we get $S_k^i\approx_{r-1} S_k^j$, and so after the duplicator's move, the two structures are $\approx_{r-1}$-isomorphic, meaning that the duplicator has a winning strategy for a $r-1$-rounds game on the current structures. This concludes the proof of fact 2.
\end{slide}

\begin{slide}
\begin{center}
{%\large
\color{blue}
Back to the proof
}
\\
\_\hrulefill
\end{center}
\small

Recall that we want to prove that $\overline{\mu}(\varphi) = \lim_{x\to R}\frac{c(x)}{a(x)}$ always exists, for every MSOL formula $\varphi$.

Let us consider the $\approx_{r}$-classes of connected structures in $C$, $D_0,\dots,D_{l-1}$, with egf $c_0(x),\dots,c_{l-1}(x)$, so $c(x) = \sum c_i(x)$.

Recall fact 1:

\emph{Fact 1}: If $\mathcal{M}_0\approx_r\mathcal{N}_0$ and $\mathcal{M}_1\approx_r\mathcal{N}_1$ then $\mathcal{M}_0\uplus\mathcal{M}_1\approx_r\mathcal{N}_0\uplus\mathcal{N}_1$

This fact implies that to know what MSOL sentences  of quantifier rank $r$ hold in some structure, it suffices to know how many components that structure has in all the $D_i$'s 
(since we can think of the structure as built by a disjoint union of all those components).
\end{slide}

\begin{slide}
\begin{center}
{%\large
\color{blue}
Back to the proof (cont.)
}
\\
\_\hrulefill
\end{center}
\small

Now recall fact 2:

Let us consider the $\approx_{r}$-classes of connected structures in $C$, $D_0,\dots,D_{l-1}$, with egf $c_0(x),\dots,c_{l-1}(x)$, so $c(x) = \sum c_i(x)$.

\emph{Fact 2}: For every $r$ and a similarity type there is an integer $s$ such that whenever $i,j\ge s$ and $\mathcal{M}$ is a structure of this similarity type, $i\cdot\mathcal{M}\approx_rj\cdot\mathcal{M}$

This implies we don't really need to know the exact number of components in $D_i$ if it exceeds $s$; two structures with $s$ or more components in $D_i$ are isomorphic as far as $D_i$ is concerned.

Thus, every structure in $C$ can be charactarized by a ''signature`` which is a tuple $\alpha = (j_0,j_1,\dots,j_{l-1})$ of the number of components that structure has in the equivalence classes $D_0,\dots,D_{l-1}$. Two structures with the same signature are $\approx_{r}$ equivalent.

So, if two structures in $C$ have the same signatures, either both of them satisfy $\varphi$, or both of them does not satisfy $\varphi$. 

For a signature $\alpha$, denote by $c_\alpha(x)$ the egf corresponding to the class of structures with signature $\alpha$.

And so, the egf for the set of structures satisfying $\varphi$ is $\sum_\alpha c_\alpha(x)$ for some subset of the signatures $\alpha$. Since this is a finite sum, we can deal with every summand by itself.

\end{slide}

\begin{slide}
\begin{center}
{%\large
\color{blue}
Converging on the goal
}
\\
\_\hrulefill
\end{center}
\small

Let us state what remains to be proved:

Given $c_\alpha(x)$, the egf for the class of structures satisfying $\alpha$, and given $a(x)$, the egf for all structures in $C$, we wish to prove that $\lim_{x\to R}\frac{c_\alpha(x)}{a(x)}$ converges.

How does $c_\alpha(x)$ look? We use the egf's $c_0(x),\dots,c_{l-1}(x)$ of the classes $D_0,\dots,D_{l-1}$.

A structure with signature $(j_0,j_1,\dots,j_{l-1}$ is built by taking $j_0$ components from $D_0$, $j_1$ components from $D_1$, etc. and performing a disjoint union on the components selected. 
One exception is that when $j_i = s$ we do not have to take exactly $s$ components, but any number $\ge s$ of components.

In general, if we have two classes of component with egf's $A(x),B(x)$ and we take one component from each class and perform a disjoint union, the egf of all
elements obtained in this way is $A(x)\cdot B(x)$.

Similarily, if we take $j$ elements from the class with egf $A(x)$, our egf will be $A(x)^j/j!$ (we divide by $j!$ since the order in which we choose 
\end{slide}


\end{document}
