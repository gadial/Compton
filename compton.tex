\documentclass[landscape,a4]{myslides}
\usepackage{theorem}
\usepackage[dvips]{color}
\usepackage{calc}
\usepackage{ifthen}
\usepackage{epsfig}
%\usepackage{graphicx}
\usepackage{latexsym}
\usepackage{amssymb,amsmath}
\input{smacros}
\skiptrue
%----------------headers----------------------
\def\Lecturename{Logical Methods in Combinatorics, 236605-2009/10}  %xxx=ESSLLI'99
\def\Lecture{Lecture ?}      %xxx=Lecture

\begin{document}

\begin{slide}
\begin{center}
{%\large
\color{blue}
0-1 laws for slow growing classes
}
\\
\_\hrulefill
\end{center}

Our goal:
{\small
\begin{itemize}
\item
Define ``slow growing'' classes of structures (as opposed to ``fast growing'' which we shall not discuss)
\item
Find a probability measure on MSOL sentences that extends the usual measure (but is defined for any MSOL sentence)
\item
Study under which conditions a class has a 0-1 law with respect to this measure.
\end{itemize}
}
\end{slide}
 
\begin{slide}
\begin{center}
{%\large
% 17 lines
\color{blue}
Definitions
}
\\
\_\hrulefill

\end{center}
\small
\begin{itemize}
 \item Let $C$ be a class of labeled structures, and $C_n$ the class of structures of size $n$ in $C$.
 \item Set $a_n = |C_n|$
 \item The \emph{exponential generating series} for $C$ is the formal power series $a(x)=\sum_{n=0}^\infty\frac{a_n}{n!}x^n$
 \item When considering $a(x)$ as an analytical function on $\mathbb{C}$ it makes sense to consider its radius of convergence $R$. If $R>0$ we say that $C$ is
a \emph{slow growing} class (since the faster the sequence $a_n$ grows, the smaller the radius of convergence).
\end{itemize}

\end{slide}

\begin{slide}
\begin{center}
{%\large
\color{blue}
The problem with the standard probability measure
}
\\
\_\hrulefill


\end{center}

\small
Recall that given a  sentence $\varphi$ we defined $\mu(\varphi)=\lim_{n\to\infty}\mu_n(\varphi)$ with $\mu_n(\varphi)=|\left\{\mathcal{M}\in C_n|\mathcal{M}\models\varphi \right\}|$.

However, the value of $\mu_n(\varphi)$ might constantly alternate values, causing $\mu(\varphi)$ to be undefined.

Example:

Let $\varphi$ be the following property of $G=(V,E)$: ``$|V|$ is even and $|E|=0$'' (expressable in SOL).

Obviously, $\mu_{2n}(\varphi) = 1$ and $\mu_{2n+1}(\varphi) = 0$
\end{slide}
\end{document}
